\chapter*{Introduction}
Un thésaurus est un type de langage documentaire qui constitue un vocabulaire normalisé. Il regroupe de manière organisée les termes d'un même domaine de connaissance. Cet outil linguistique permet de décrire des concepts et de lever les ambiguïtés induites par les relations de synonymie, d'homonymie et de polysémie présentes dans le langage naturel.

L'outil développé lors de ce projet sera composé de termes décrivant des concepts, reliés entre eux par des relations hiérarchiques, synonymiques et associatives. L'utilisateur aura la possibilité d'explorer la hiérarchie et de gérer (ajouter / modifier / supprimer) les termes et les concepts.

Ce travail, réalisé par une équipe de cinq étudiants\footnote{Baptiste Le Bail, Thibaut Marmin, Namrata Patel, Clément Sipieter, Steeve Tuvée}, est présenté dans ce rapport selon trois phases distinctes : analyse, conception, implémentation.