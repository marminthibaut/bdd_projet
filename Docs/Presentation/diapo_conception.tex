\subsection{Paradigmes}
\begin{frame}{Paradigmes}{Vision objet}
  \begin{block}{Avantages}
    \begin{itemize}
      \item abstraction
      \item réutilisation
      \item maintenance
    \end{itemize}
  \end{block}
  \begin{block}{Inconvenients}
    \begin{itemize}
      \item peu de ODBMS
    \end{itemize}
  \end{block}
\end{frame}

\begin{frame}{Paradigmes}{Vision objet-relationnel}
  \begin{block}{Avantages}
    \begin{itemize}
      \item norme SQL3
      \item accès direct par pointeurs
    \end{itemize}
  \end{block}
  \begin{block}{Inconvenients}
    \begin{itemize}
      \item peu implémenté par les DBMS
      \item différences entre la norme et les implémentations
      \item manque de généricité
    \end{itemize}
  \end{block}
\end{frame}

\begin{frame}{Paradigmes}{Vision relationnel pur}
  \begin{block}{Avantages}
    \begin{itemize}
      \item mature
      \item performant
      \item généricité
    \end{itemize}
  \end{block}
  \begin{block}{Inconvenients}
    \begin{itemize}
      \item schéma relationnel à adapter à une programmation souvent objet
    \end{itemize}
  \end{block}
\end{frame}

\subsection{ORM}
\begin{frame}{ORM}{Qu'est ce qu'un ORM ?}
  \begin{block}{Object relational-mapping}
    \begin{itemize}
      \item composant logiciel
      \item interface entre application objet et BD relationnel
      \item illusion d'une BD objet
    \end{itemize}
  \end{block}
\end{frame}

\begin{frame}{ORM}{Pourquoi ?}
  \begin{itemize}
    \item schéma de données unique
    \item abstraction de la BD
    \item automatisation de la persistance des données
  \end{itemize}
\end{frame}

\subsection{Décision}
\begin{frame}{Décision}
  \begin{itemize}
    \item PostgresQL
    \item ORM
  \end{itemize}
\end{frame}
